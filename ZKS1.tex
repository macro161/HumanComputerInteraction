\documentclass[oneside]{VUMIFPSkursinis}
\usepackage{algorithmicx}
\usepackage{algorithm}
\usepackage{algpseudocode}
\usepackage{amsfonts}
\usepackage{float}
\usepackage{amsmath}
\usepackage{bm}
\usepackage{caption}
\usepackage{color}
\usepackage{float}
\usepackage{graphicx}
\usepackage{listings}
\usepackage{subfig}
\usepackage{ltablex}
\usepackage{longtable}
\usepackage{wrapfig}
\usepackage{subfig}
\usepackage{pbox}
\renewcommand{\labelenumii}{\theenumii}
\renewcommand{\theenumii}{\theenumi.\arabic{enumii}.}
\renewcommand{\labelenumiii}{\theenumiii}
\renewcommand{\theenumiii}{\theenumii\arabic{enumiii}.}
\newcolumntype{P}[1]{>{\centering\arraybackslash}p{#1}}
\usepackage[%
	colorlinks=true,
	linkcolor=black
]{hyperref}
\university{Vilniaus universitetas}
\faculty{Matematikos ir informatikos fakultetas}
\department{Programų sistemų katedra}
\papertype{Žmogaus ir kompiuterio sąsaja laboratorinis darbas I}
\title{Alaus gamybos įrangos ir ingridientų pirkimo sistema}
\titleineng{Aludarystės internetinė parduotuvė}
\status{3 kurso studentai}
\author{Greta Pyrantaitė}
\secondauthor{Matas Savickis}
\thirdauthor{Andrius Bentkaus}

\supervisor{Kristina Lapin, Doc., Dr.}
\date{Vilnius – \the\year}

\bibliography{bibliografija}

\begin{document}
\maketitle

\section{Anotacija}
Šiuo darbu siekiama išanalizuoti ir aprašyti dabartinės www.savasalus.lt sąsajos napatogumus, paaiškinti koks panaudojimo principas buvo pažeistas ir šio pažeidimo priežastis.
Šiame darbe taip pat atkreipsime dėmesį į sistemos vartotojų grupes ir jų sąveika su sitema.
Darbo eigoje apžvelgsime pasisekusias vartotojo sąsajos realizacijas ir aptarsime, kodėl būtent tokie sprendimai yra geresni už esamos sistemos sąsajų realizacijas.

\begin{itemize}
	\item{Greta Pyrantaitė - Greta.pyrantaite@gmail.com}
	\item{Matas Savickis - savickismatas@gmail.com}
	\item{Andrius Bentkus - andrius.bentkus@gmail.com}
\end{itemize}

\tableofcontents

\section{Įvadas}
	\subsection{Programų sistemos pavadinimas}
		Alaus gamybos įrangos ir ingridientų pirkimo sisema
	\subsection{Dalykinė sritis}
		Elektroninė parduotuvė skirta aludariams
	\subsection{Probleminė sritis}
		Sistema turi suteikt galimybę nusipirkti alaus gamybai reikalingus ingridientus ir įrangą bei gauti visą reikalingą informaciją apie ingridientų ir įrangos speficikacijas.
		Sistema taip pat turi vartotojui pateikti informaciją apie alaus gamybos procesą panaudojant nusipirktus ingridientus ir įrangą.
	\subsection{Naudotojai}
		\begin{itemize}
			\item{Pirkėjas(aludaris) - pirkėjas turi galėti užsisakyti alaus gamybos ingridientus ir įrangą bei gauti reikalingą specifikaciją norint naudotis prekia.
				Pirkėjui turi užtekti mokyklinio informatikos kurso žinių ir bendro supratimo kaip naviguotis internetinėse svetainėse.}
			\item{Pardavėjas - pardavėjui sistema turi suteikti informaciją apie užsakymus, jų apmokėjimus ir panašia svarbią informaciją.
				Sistema pardavėjui taip pat turi suteikti galimybė pridėti arba išimti prekes iš internetinės parduotuvės.
				Pardavėjui taip pat turi užtekti mokyklinio lygio informatikos žinių ir bendrų žinių naviguotis interneto svetainėse.}
		\end{itemize}
	\subsection{Darbo pagrindas}
	\subsection{Naudoti dokumentai}

\section{Būsimos sistemos įtakojamų asmenų kategorijos}
\section{Naudotojų charakteristos}

\section{Pirmą kartą alaus gamyba užsiemančio pirkėjo poreikiai}
	\subsection{Naudotojų charakteristikos}
		Naudotojas pirmą karta bando išvirti alų.
	\subsection{Kompiuterizuojamos veiklos analizė}
		Petrui yra 19 metų ir jis bando išsivirti alų dėl neseniai įvesto Lietuvos sauso įstatymo.
		Petras žino jog alkoholio akcizas šiuo metu yra labai aukštas ir savo sunkiai uždirbtus pinigus išleisti perkant produktus turinčius alkoholį yra ne pats efektyviausias iš ekonomiško taško žiūrint.
		Jis nueina į aludarystės internetinę parduotuvę ir bando gauti informaciją kaip pirmą kartą pasigaminti alaus.
		Ši informacija nėra lengvai pateikta puslapyje ir Petras ilgai ieško jos, tol kol susiranda pradinuko alaus gamintojo komplektą, kuriame kažkur produkto aprašyme užslėptai yra pridėtas instruktažas.
		Petras įsideda šį komplekta į krepšelį ir neuina į atsiskaitymo formą, kurioje užildo reikiama informaciją ir susimoka.
	\subsection{Panaudojamumo siekiai ir matai}
		Vienas iš svarbiausių siekių yra sumažinti laiko tarpą, per kurį vidutiniškai visiškas alaus gamybos pradinukas ateina į puslapį, išsaiškina, ko reikia nupirkti alaus gamybai ir susimokėjima už prekes.
		Kuo lengvesnis ir aiškesnis šitas procesas būs, tuo didesnė tikimybė, kad vartotojas nusipirks pirmą alaus gamybos komplekta ir taip pat padidės galimybė, jog vartotojas grįž atgal į parduotuvę.
		Pagrindinis matas yra laiko tarpas, kurio reikia naujam vartotjui pirmą kartą atėjus į puslapį ir nusipirkti pirmą alaus gamybos komplektą.
		Siekis yra pateikti kuo intuityviau pradedančiui aludariui informaciją kaip pasigaminti alų ir to pasekoje sumažinti jam reikiamą laiką tarp atėjimo į puslapį ir pirmo komplekto nusipirkimo.

\section{Patyrusio aludario pirkėjo poreikiai}
	\subsection{Naudotojų charakteristikos}
		Naudotojas ieško specifinių ingridientų ir įrangos pagerinti alaus galybos procesą ir gaminio kokybę.
	\subsection{Kompiuterizuojamos veiklos analizė}
		Matas yra patyręs aludaris siekiantis įsigyti specialios įrangos ir ingridientų pagerinti alaus gamybos procesą ir pagaminto produkto kokybę.
		Dabartinėje sistemoje ingridientų paieška yra labai paviršutiniška.
		Pavyzdžiui norėdamas surasti tam tikro rūgštingumo apynių Matas turi atsidaryti apynių skiltį pagrindiniame puslapyje, atsidariusiame naujame lange jis pasirenkas apynių tipą.
		Atsidaro langas su visomis paruotuvės turimomis prekėmis.
		Parduotuvėje produktu paieškos pagal tam tikrus parametus nėra, o ir prekių sąraše informacija nenurodyta, todėl Matas, norėdamas surasti tam tikros specifikacijos mielių turi spausti ant kiekvienos prekės ir skaityti jos aprašyma.
		Tai yra labai nepatogu ir užimas daug laiko.
		Net ir suradus reikiamą produktą neparodomos alternatyvos duotam produktui.
		Tokia pati problema vyrauja ir mielių, salyklo, įrangos bei kitų prekių paieškoje.

	\subsubsection{Problemos ir neišnaudotos galimybės}
		\begin{itemize}
			\item{Prekės paieška yra rezultatyvi tuo atveju jeigu vartotojas tiksliai žino ko ieško ir pereina per visas parduotuvėje siūlomas prekes}
			\item{Naudotojui trūksta paieškos galimybių}
			\item{Pasirinkus prekę vartotojui nepasiūlomos alternatyvos tai prekiai(pvz. pigesnis kitos gamintojo variantas)}
		\end{itemize}
	\subsubsection{Veiklos dažniai}
		Sistema vartotojas naudojasi sistema 1-2 kartus per mėnesį.
		Šio dažnio vartotojas pats žino ko ieškoti todėl jam nereikia prisiminti kaip kiekvieną kartą vykdyti paiešką.
		Vartotojas turėtų surasti reikiamą prekią per 5 minutes.
		Kiekvienas vartotojas individualiai žino kokių prekių jam reikia.

\section{Panaudojamumo siekiai ir matai}

\section{Būsimasis scenarijus}

\section{Įkvepiančios esamų interfeisų idėjos}

\section{Žodynas}

\end{document}
