\documentclass[oneside]{VUMIFPSkursinis}
\usepackage{algorithmicx}
\usepackage{algorithm}
\usepackage{algpseudocode}
\usepackage{amsfonts}
\usepackage{float}
\usepackage{amsmath}
\usepackage{bm}
\usepackage{caption}
\usepackage{color}
\usepackage{float}
\usepackage{graphicx}
\usepackage{listings}
\usepackage{subfig}
\usepackage{ltablex}
\usepackage{longtable}
\usepackage{wrapfig}
\usepackage{subfig}
\usepackage{pbox}
\renewcommand{\labelenumii}{\theenumii}
\renewcommand{\theenumii}{\theenumi.\arabic{enumii}.}
\renewcommand{\labelenumiii}{\theenumiii}
\renewcommand{\theenumiii}{\theenumii\arabic{enumiii}.}
\newcolumntype{P}[1]{>{\centering\arraybackslash}p{#1}}
\usepackage[%  
    colorlinks=true,
    linkcolor=black
]{hyperref}
\university{Vilniaus universitetas}
\faculty{Matematikos ir informatikos fakultetas}
\department{Programų sistemų katedra}
\papertype{Žmogaus ir kompiuterio sąsaja laboratorinis darbas I}
\title{Alaus gamybos įrangos ir ingridientų pirkimo sistema}
\titleineng{Aludarystės sistema}
\status{3 kurso studentai}
\author{Greta Pyrantaitė}
\secondauthor{Matas Savickis}  
\thirdauthor{Andrius Bentkaus}   

\supervisor{Kristina Lapin, Doc., Dr.}
\date{Vilnius – \the\year}

\bibliography{bibliografija}

\begin{document}
\maketitle

\section{Anotacija}
Šiuo darbu siekiama išanalizuoti ir aprašyti dabartinės www.savasalus.lt sąsajos napatogumus, paaiškinti koks panaudojimo principas buvo pažeistas ir šio pažeidimo priežastis.
Šiame darbe taip pat atkreipsime dėmesį į sistemos vartotojų grupes ir jų sąveika su sitema.
Darbo eigoje apžvelgsime pasisekusias vartotojo sąsajos realizacijas ir aptarsime, kodėl būtent tokie sprendimai yra geresni už esamos sistemos sąsajų realizacijas.

\begin{itemize}
	\item{Greta Pyrantaitė - }
	\item{Matas Savickis - savickismatas@gmail.com}
	\item{Andrius Bentkus - andrius.bentkus@gmail.com}
\end{itemize}

\tableofcontents

\section{Įvadas}
	\subsection{Programų sistemos pavadinimas}
		Alaus gamybos įrangos ir ingridientų pirkimo sisema
	\subsection{Dalykinė sritis}
		Elektroninė parduotuvė skirta aludariams 
	\subsection{Probleminė sritis}
		Sistema turi suteikt galimybę nusipirkti alaus gamybai reikalingus ingridientus ir įrangą bei gauti visą reikalingą informaciją apie ingridientų ir įrangos speficikacijas. 
Sistema taip pat turi vartotojui pateikti informaciją apie alaus gamybos procesą panaudojant nusipirktus ingridientus ir įrangą.
	\subsection{Naudotojai}
	\subsection{Darbo pagrindas}
	\subsection{Naudoti dokumentai}

\section{Pirmos naudotojų grupės poreikiai}
	\subsection{Naudotojų charakteristikos}
	\subsection{Kompiuterizuojamos veiklos analizė}
	\subsection{Panaudojamumo siekiai ir matai}

\section{Įkvepiančios esamų interfeisų idėjos}

\section{Žodynas}

\end{document}