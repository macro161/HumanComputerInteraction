\documentclass[oneside]{VUMIFPSkursinis}
\usepackage{algorithmicx}
\usepackage{algorithm}
\usepackage{algpseudocode}
\usepackage{amsfonts}
\usepackage{float}
\usepackage{amsmath}
\usepackage{bm}
\usepackage{caption}
\usepackage{color}
\usepackage{float}
\usepackage{graphicx}
\usepackage{listings}
\usepackage{subfig}
\usepackage{ltablex}
\usepackage{longtable}
\usepackage{wrapfig}
\usepackage{subfig}
\usepackage{pbox}
\renewcommand{\labelenumii}{\theenumii}
\renewcommand{\theenumii}{\theenumi.\arabic{enumii}.}
\renewcommand{\labelenumiii}{\theenumiii}
\renewcommand{\theenumiii}{\theenumii\arabic{enumiii}.}
\newcolumntype{P}[1]{>{\centering\arraybackslash}p{#1}}
\usepackage[%
	colorlinks=true,
	linkcolor=black
]{hyperref}
\university{Vilniaus universitetas}
\faculty{Matematikos ir informatikos fakultetas}
\department{Programų sistemų katedra}
\papertype{Žmogaus ir kompiuterio sąsaja laboratorinis darbas I}
\title{Alaus gamybos įrangos ir ingridientų pirkimo sistema}
\titleineng{Aludarystės internetinė parduotuvė}
\status{3 kurso studentai}
\author{Greta Pyrantaitė}
\secondauthor{Matas Savickis}
\thirdauthor{Andrius Bentkaus}

\supervisor{Kristina Lapin, Doc., Dr.}
\date{Vilnius – \the\year}

\bibliography{bibliografija}

\begin{document}
\maketitle

\section{Anotacija}
Šiuo darbu siekiama išanalizuoti ir aprašyti dabartinės www.savasalus.lt sąsajos napatogumus, paaiškinti koks panaudojimo principas buvo pažeistas ir šio pažeidimo priežastis.
Šiame darbe taip pat atkreipsime dėmesį į sistemos vartotojų grupes ir jų sąveika su sitema.
Darbo eigoje apžvelgsime pasisekusias vartotojo sąsajos realizacijas ir aptarsime, kodėl būtent tokie sprendimai yra geresni už esamos sistemos sąsajų realizacijas.

\begin{itemize}
	\item{Greta Pyrantaitė - greta.pyrantaite@gmail.com}
	\item{Matas Savickis - savickis.matas@gmail.com}
	\item{Andrius Bentkus - andrius.bentkus@gmail.com}
\end{itemize}

\tableofcontents

\section{Įvadas}
	\subsection{Programų sistemos pavadinimas}
		Alaus gamybos įrangos ir ingridientų pirkimo sisema
	\subsection{Dalykinė sritis}
		Elektroninė parduotuvė skirta aludariams
	\subsection{Probleminė sritis}
		Sistema turi suteikt galimybę nusipirkti alaus gamybai reikalingus ingridientus ir įrangą bei gauti visą reikalingą informaciją apie ingridientų ir įrangos speficikacijas.
		Sistema taip pat turi vartotojui pateikti informaciją apie alaus gamybos procesą panaudojant nusipirktus ingridientus ir įrangą.
	\subsection{Naudotojai}
		\begin{itemize}
			\item{Pirkėjas(aludaris) - pirkėjas turi galėti užsisakyti alaus gamybos ingridientus ir įrangą bei gauti reikalingą specifikaciją norint naudotis prekia.
				Pirkėjui turi užtekti mokyklinio informatikos kurso žinių ir bendro supratimo kaip naviguotis internetinėse svetainėse.}
			\item{Pardavėjas - pardavėjui sistema turi suteikti informaciją apie užsakymus, jų apmokėjimus ir panašia svarbią informaciją.
				Sistema pardavėjui taip pat turi suteikti galimybė pridėti arba išimti prekes iš internetinės parduotuvės.
				Pardavėjui taip pat turi užtekti mokyklinio lygio informatikos žinių ir bendrų žinių naviguotis interneto svetainėse.}
		\end{itemize}
	\subsection{Darbo pagrindas}
	\subsection{Naudoti dokumentai}

\section{Būsimos sistemos įtakojamų asmenų kategorijos}
	\subsection{Suinteresuotų asmenų grupės}
		\begin{itemize}
			\item{Pirminiai - pirkėjas perkantis prekes iš parduotuvės, pardavėjęs siekiantis užtikrinti prekių pasilūlą ir prekių pristatymą pirkėjui}
			\item{Antriniai - parduotuvės savininkas, kurio pelnas priklausys nuo žmonių besinaudojančių jo parduotuve skaičiaus.}
			\item{Tretiniai - kokurentai (www.aluteksas.lt) kurių pelną ir vartotojų kiekį įtakos mūsų sistemos sėkmė arba nesekmė. }
			\item{Aptarnaujantieji - programuotojai, kuriems teks palaikyti ir ateityje plėsti programų sistemą}
		\end{itemize}
\section{Naudotojų charakteristos}
Šioje skirtyje pateiksime potencialių parduotuvės naudotojų charakterisikas.

\begin{center}
\begin{tabular}{ | m{7em} | m{15cm}| }
\hline
Naudojamos IT & Moderni interneto naršyklė, išmanieji telefonai, planšetės, nešiojami kompiuteriai, stacionarūs kompiuteriai  \\
\hline
Įgudžiai, motyvacija naudotis IT & Vartotojai turintys skirtingus informacinių technologijų žinias. Vartotojai gali būti su skirtingais fiziniais pajėgumais.
Gali pasitaikyti vartotojų, kurie parduotuve naudotūsi aukštos drėgmės aplinkoje ir jiems būtų sunkiau naudotis liečiamuju ekranu.  \\
\hline
Veiklų kontekstas & Vidutinė galimybė gauti pagalbą.
Vartotojas gali susisiekti su parduotuvės personu ir pasikonsultuoti dėl pirkinių tačiau bendru atveju vartotojas atsakymo negaus iškarto.
Kiekvieno vartotojo poreikiai yra skirtingi todėl visa bazinė informacija turi būti pateikta parduotuvės puslapyje.
Svarbu užtikrinti prieiga asmenims su regos ir motorinėmiais sunkumais. \\
\hline
\end{tabular}
\end{center}
\section{Pradedančio aludario pirkėjo porekiai}
	\subsection{Naudotojų charakteristikos}
		Naudotojas pirmą kartą bando išvirti alų, jam reikalingas instruktažas.
	\subsection{Kompiuterizuojamos veiklos analizė}
		Petrui yra 19 metų ir jis bando išsivirti alų dėl neseniai įvesto Lietuvos sauso įstatymo.
		Petras žino jog alkoholio akcizas šiuo metu yra labai aukštas ir savo sunkiai uždirbtus pinigus išleisti perkant produktus turinčius alkoholį yra ne pats efektyviausias iš ekonomiško taško žiūrint.
		Jis nueina į aludarystės internetinę parduotuvę ir bando gauti informaciją kaip pirmą kartą pasigaminti alaus.
		Ši informacija nėra lengvai pateikta puslapyje ir Petras ilgai ieško jos, tol kol susiranda pradinuko alaus gamintojo komplektą, kuriame kažkur produkto aprašyme užslėptai yra pridėtas instruktažas.
		Petras įsideda šį komplekta į krepšelį ir neuina į atsiskaitymo formą, kurioje užpildo reikiamą informaciją ir susimoka.
	\subsubsection{Problemos ir neišnaudotos galimybės}
		\begin{itemize}
			\item{Pirminė informacija pradedančiam aludariui yra sunkiai surandama ir neaiškiai pateikta}
			\item{Pirminę informaciją įsisavinti ir nusipirkti tinkamus produktus atitinkančius ją yra netriviali užduotis ir svetainė jokiais aspektais nepalengvina šią užduotį}
			\item{Įgyvenditi šį scenarijų užtrunka ilgiau negu vidutinis vartotojas turi kantrybės}
		\end{itemize}
	\subsubsection{Veiklos dažniai}
		Sistemos vartotojas pirmą kartą prisijungti gali tiktais vieną kartą, todėl ši veikla nėra pasikartojanti.
		Tačiau šioje veikloje svarbu, kad vartotojas praeitų visus žingsnius sėkmingai, tam kad įvertintų sistemą tinkama tolimesniam naudojamui, kitų veiklų.

\section{Patyrusio aludario pirkėjo poreikiai}
	\subsection{Naudotojų charakteristikos}
		Naudotojas ieško specifinių ingridientų ir įrangos pagerinti alaus galybos procesą ir gaminio kokybę.
	\subsection{Kompiuterizuojamos veiklos analizė}
		Matas yra patyręs aludaris siekiantis įsigyti specialios įrangos ir ingridientų pagerinti alaus gamybos procesą ir pagaminto produkto kokybę.
		Dabartinėje sistemoje ingridientų paieška yra labai paviršutiniška.
		Pavyzdžiui norėdamas surasti tam tikro rūgštingumo apynių Matas turi atsidaryti apynių skiltį pagrindiniame puslapyje, atsidariusiame naujame lange jis pasirenkas apynių tipą.
		Atsidaro langas su visomis su visom prekėmis, kurios yra parduodamas parduotuvėje.
		Parduotuvėje produktų paieškos pagal tam tikrus parametus nėra, taip pat prekių sąraše informacija nenurodyta, todėl Matas, norėdamas surasti tam tikros specifikacijos mielių turi peržiūrėti kiekviną prekę ir skaityti jos aprašymą.
		Tai yra labai nepatogus, daug laiko užimantis ir varginantis procesas.
		Net ir suradus reikiamą produktą neparodomos alternatyvos duotam produktui.
		Tokia pati problema vyrauja ir mielių, salyklo, įrangos bei kitų prekių paieškoje.
	\subsubsection{Problemos ir neišnaudotos galimybės}
		\begin{itemize}
			\item{Prekės paieška yra rezultatyvi tuo atveju jeigu vartotojas tiksliai žino ko ieško ir pereina per visas parduotuvėje siūlomas prekes}
			\item{Naudotojui trūksta paieškos galimybių}
			\item{Vartotojui pasirinkus prekę sistema nepasiūlo alternatyvos tai prekei(pvz. pigesnis kitos gamintojo variantas)}
		\end{itemize}
	\subsubsection{Veiklos dažniai}
		Sistema vartotojas naudojasi sistema 1-2 kartus per mėnesį.
		Šio scenarijaus vartotojas pats žino ko ieškoti todėl jam nereikia prisiminti kaip kiekvieną kartą vykdyti paiešką.
		Vartotojas turėtų surasti reikiamą prekę per 5 minutes.
		Kiekvienas vartotojas individualiai žino kokių prekių jam reikia.

\section{Panaudojamumo siekiai ir matai}

\section{Būsimasis scenarijus}

\section{Įkvepiančios esamų interfeisų idėjos}

\section{Žodynas}

\end{document}
