\documentclass[oneside]{VUMIFPSkursinis}
\usepackage{algorithmicx}
\usepackage{algorithm}
\usepackage{algpseudocode}
\usepackage{amsfonts}
\usepackage{float}
\usepackage{amsmath}
\usepackage{bm}
\usepackage{caption}
\usepackage{color}
\usepackage{float}
\usepackage{graphicx}
\usepackage{listings}
\usepackage{subfig}
\usepackage{ltablex}
\usepackage{longtable}
\usepackage{wrapfig}
\usepackage{subfig}
\usepackage{pbox}
\renewcommand{\labelenumii}{\theenumii}
\renewcommand{\theenumii}{\theenumi.\arabic{enumii}.}
\renewcommand{\labelenumiii}{\theenumiii}
\renewcommand{\theenumiii}{\theenumii\arabic{enumiii}.}
\newcolumntype{P}[1]{>{\centering\arraybackslash}p{#1}}
\usepackage[%
	colorlinks=true,
	linkcolor=black
]{hyperref}
\university{Vilniaus universitetas}
\faculty{Matematikos ir informatikos fakultetas}
\department{Programų sistemų katedra}
\papertype{Žmogaus ir kompiuterio sąsaja laboratorinis darbas I}
\title{Alaus gamybos įrangos ir ingredientų pirkimo sistema}
\titleineng{Aludarystės internetinė parduotuvė}
\status{3 kurso studentai}
\author{Greta Pyrantaitė}
\secondauthor{Matas Savickis}
\thirdauthor{Andrius Bentkus}

\supervisor{Kristina Lapin, Doc., Dr.}
\date{Vilnius – \the\year}

\bibliography{bibliografija}

\begin{document}
\maketitle

\sectionnonum{Anotacija}
Šiuo darbu siekiama išanalizuoti ir aprašyti dabartinės www.savasalus.lt sąsajos napatogumus, paaiškinti, koks panaudojimo principas buvo pažeistas, ir šio pažeidimo priežastis.
Šiame darbe taip pat atkreipsime dėmesį į sistemos vartotojų grupes ir jų sąveiką su sitema.
Darbo eigoje apžvelgsime pasisekusias vartotojo sąsajos realizacijas ir aptarsime, kodėl būtent tokie sprendimai yra geresni už esamos sistemos sąsajų realizacijas.

\begin{itemize}
	\item{Greta Pyrantaitė - greta.pyrantaite@gmail.com}
	\item{Matas Savickis - savickis.matas@gmail.com}
	\item{Andrius Bentkus - andrius.bentkus@gmail.com}
\end{itemize}

\tableofcontents

\end{document}