\documentclass[oneside]{VUMIFPSkursinis}
\usepackage{algorithmicx}
\usepackage{algorithm}
\usepackage{algpseudocode}
\usepackage{amsfonts}
\usepackage{float}
\usepackage{amsmath}
\usepackage{bm}
\usepackage{caption}
\usepackage{color}
\usepackage{float}
\usepackage{graphicx}
\usepackage{listings}
\usepackage{subfig}
\usepackage{ltablex}
\usepackage{longtable}
\usepackage{wrapfig}
\usepackage{subfig}
\usepackage{pbox}
\renewcommand{\labelenumii}{\theenumii}
\renewcommand{\theenumii}{\theenumi.\arabic{enumii}.}
\renewcommand{\labelenumiii}{\theenumiii}
\renewcommand{\theenumiii}{\theenumii\arabic{enumiii}.}
\newcolumntype{P}[1]{>{\centering\arraybackslash}p{#1}}
\usepackage[%
	colorlinks=true,
	linkcolor=black
]{hyperref}
\university{Vilniaus universitetas}
\faculty{Matematikos ir informatikos fakultetas}
\department{Programų sistemų katedra}
\papertype{Žmogaus ir kompiuterio sąsaja laboratorinis darbas I}
\title{Alaus gamybos įrangos ir ingridientų pirkimo sistema}
\titleineng{Aludarystės internetinė parduotuvė}
\status{3 kurso studentai}
\author{Greta Pyrantaitė}
\secondauthor{Matas Savickis}
\thirdauthor{Andrius Bentkus}

\supervisor{Kristina Lapin, Doc., Dr.}
\date{Vilnius – \the\year}

\bibliography{bibliografija}

\begin{document}
\maketitle

\section{Anotacija}
Šiuo darbu siekiama išanalizuoti ir aprašyti dabartinės www.savasalus.lt sąsajos napatogumus, paaiškinti, koks panaudojimo principas buvo pažeistas, ir šio pažeidimo priežastis.
Šiame darbe taip pat atkreipsime dėmesį į sistemos vartotojų grupes ir jų sąveiką su sitema.
Darbo eigoje apžvelgsime pasisekusias vartotojo sąsajos realizacijas ir aptarsime, kodėl būtent tokie sprendimai yra geresni už esamos sistemos sąsajų realizacijas.

\begin{itemize}
	\item{Greta Pyrantaitė - greta.pyrantaite@gmail.com}
	\item{Matas Savickis - savickis.matas@gmail.com}
	\item{Andrius Bentkus - andrius.bentkus@gmail.com}
\end{itemize}

\tableofcontents

\section{Įvadas}
	\subsection{Programų sistemos pavadinimas}
		Alaus gamybos įrangos ir ingridientų pirkimo sisema
	\subsection{Dalykinė sritis}
		Elektroninė parduotuvė skirta aludariams
	\subsection{Probleminė sritis}
		Sistema turi suteikt galimybę nusipirkti alaus gamybai reikalingus ingridientus ir įrangą bei gauti visą reikalingą informaciją apie ingridientų ir įrangos specifikacijas.
		Sistema taip pat turi vartotojui pateikti informaciją apie alaus gamybos procesą panaudojant nusipirktus ingridientus ir įrangą.
	\subsection{Naudotojai}
		\begin{itemize}
			\item{Pirkėjas(aludaris) - pirkėjas turi galėti užsisakyti alaus gamybos ingridientus ir įrangą bei gauti reikalingą specifikaciją norint naudotis preke.
				Pirkėjui turi užtekti mokyklinio informatikos kurso žinių ir bendro supratimo, kaip naviguotis internetinėse svetainėse.}
			\item{Pardavėjas - pardavėjui sistema turi suteikti informaciją apie užsakymus, jų apmokėjimus ir panašią svarbią informaciją.
				Sistema pardavėjui taip pat turi suteikti galimybę pridėti arba išimti prekes iš internetinės parduotuvės.
				Pardavėjui taip pat turi užtekti mokyklinio lygio informatikos žinių ir bendrų žinių naviguotis interneto svetainėse.}
		\end{itemize}
	\subsection{Darbo pagrindas}
	\subsection{Naudoti dokumentai}

\section{Būsimos sistemos įtakojamų asmenų kategorijos}
\begin{itemize}
			\item{Pirminiai - pirkėjas kuris tiesiogiai naudojasi sistema ir įsigija prekes iš parduotuvės. 
				Pardavėjas, kuris paruošia pirkėjo užsakymą. 
				Prekių tiekėjas, kurio pelnas didėja nuo pirkėjų skaičiaus.}
			\item{Antriniai - parduotuvės savininkas, kuris gauna statistiką apie prekes ir ruošia personalizuotus užsakymus pirkėjui.}
			\item{Tretiniai - konkurentai (www.aluteksas.lt), kurių pelną ir vartotojų kiekį įtakos mūsų sistemos sėkmė arba nesekmė. }
			\item{Aptarnaujantieji - programuotojai, kuriems teks palaikyti ir ateityje plėsti programų sistemą.}
		\end{itemize}

\section{Pirkėjų grupės poreikiai}
	\subsection{Naudotojų charakteristikos}
		\subsubsection{Informacinių technologijų priemonės} 
			Moderni interneto naršyklė, išmanieji telefonai, planšetės, nešiojami kompiuteriai, stacionarūs kompiuteriai. 
		\subsubsection{Motyvacija ir galimybės tobulintiu įgudžius}
			 Vartotojai, turintys skirtingus informacinių technologijų žinias. Vartotojai gali būti su skirtingais fiziniais pajėgumais.
			Gali pasitaikyti vartotojų, kurie parduotuve naudotųsi aukštos drėgmės aplinkoje ir jiems būtų sunkiau naudotis liečiamuoju ekranu. 
		\subsubsection{Veiklų kontekstai}
			Vidutinė galimybė gauti pagalbą.
			Vartotojas gali susisiekti su parduotuvės personalu ir pasikonsultuoti dėl pirkinių, tačiau bendru atveju vartotojas atsakymo negaus iš karto.
			Kiekvieno vartotojo poreikiai yra skirtingi, todėl visa bazinė informacija turi būti pateikta parduotuvės puslapyje.
			Svarbu užtikrinti prieigą asmenims su regos ir motoriniais sunkumais.
		\subsubsection{Naudotojų tipas}
			\begin{itemize}
				\item{Naujokas aludaris - Naudotojas pirmą kartą bando išvirti alų, jam reikalingas instruktažas.}
				\item{Patyręs aludaris - Naudotojas ieško specifinių ingridientų ir įrangos pagerinti alaus gamybos procesą ir gaminio kokybę.}
			\end{itemize}
	\subsection{Kompiuterizuojamų veiklų analizė}
		\subsubsection{Konceptinis scenarijus}
			\begin{itemize}
				\item{Naujokas aludaris - Petrui yra 19 metų ir jis bando išsivirti alų dėl neseniai įvesto Lietuvos sauso įstatymo.
					Petras žino jog alkoholio akcizas šiuo metu yra labai aukštas ir savo sunkiai uždirbtus pinigus išleisti perkant produktus, turinčius alkoholį, yra ne pats ekonomiškiausias sprendimas.
					Jis nueina į aludarystės internetinę parduotuvę ir bando gauti informaciją apie alaus gamybą namuose.
					Ši informacija nėra lengvai pateikta puslapyje ir Petras ilgai ieško jos, tol kol susiranda pradinuko alaus gamintojo komplektą, kuriame kažkur produkto aprašyme užslėptai yra pridėtas instruktažas.
					Petras įsideda šį komplektą į krepšelį ir neuina į atsiskaitymo formą, kurioje užpildo reikiamą informaciją ir susimoka.}
				\item{Patyręs aludaris - Matas yra patyręs aludaris, siekiantis įsigyti specialios įrangos ir ingridientų pagerinti alaus gamybos procesą ir pagaminti produkto kokybę.
					Dabartinėje sistemoje ingridientų paieška yra labai paviršutiniška.
					Pavyzdžiui, norėdamas surasti tam tikro rūgštingumo apynių Matas turi atsidaryti apynių skiltį pagrindiniame puslapyje, atsidariusiame naujame lange jis pasirenka apynių tipą.
					Atsidaro langas su visomis prekėmis, kurios yra parduodamos parduotuvėje.
					Parduotuvėje produktų paieškos pagal tam tikrus parametus nėra, taip pat prekių sąraše informacija nenurodyta, todėl Matas, norėdamas surasti tam tikros specifikacijos mielių turi peržiūrėti kiekviną prekę ir skaityti jos aprašymą.
					Tai yra labai nepatogus, daug laiko užimantis ir varginantis procesas.
					Net ir suradus reikiamą produktą neparodomos alternatyvos duotam produktui.
					Tokia pati problema vyrauja ir mielių, salyklo, įrangos bei kitų prekių paieškoje. }
			\end{itemize}
		\subsubsection{Veiklų charakteristikos}
			\subsubsubsection{Veiklos dažnis}
				\begin{itemize}
					\item{Naujokas aludaris - Sistemos vartotojas pirmą kartą prisijungti gali tiktais vieną kartą, todėl ši veikla nėra pasikartojanti.
						Tačiau šioje veikloje svarbu, kad vartotojas praeitų visus žingsnius sėkmingai, tam kad įvertintų sistemą tinkama tolimesniam naudojamui, kitom veiklom. }
					\item{Patyręs aludaris - vartotojas naudojasi sistema 1-2 kartus per mėnesį.
						Šiame scenarijuje vartotojas pats žino, ko ieškoti, todėl jam nereikia prisiminti, kaip kiekvieną kartą vykdyti paiešką.
						Vartotojas turėtų surasti reikiamą prekę per 5 minutes.
						Kiekvienas vartotojas individualiai žino, kokių prekių jam reikia.}
					
				\end{itemize}
		\subsubsection{Problemos ir tobulinimo galimybės}
					\begin{itemize}
						\item{Pirminė informacija pradedančiam aludariui yra sunkiai surandama ir neaiškiai pateikta}
						\item{Pirminę informaciją įsisavinti ir nusipirkti tinkamus produktus, atitinkančius ją, yra netriviali užduotis ir svetainė jokiais aspektais nepalengvina šios užduoties}
						\item{Įgyvendinti šį scenarijų užtrunka ilgiau negu vidutinis vartotojas turi kantrybės}
						\item{Prekės paieška yra rezultatyvi tuo atveju, jeigu vartotojas tiksliai žino, ko ieško, ir pereina per visas parduotuvėje siūlomas prekes}
						\item{Naudotojui trūksta paieškos galimybių}
						\item{Vartotojui pasirinkus prekę sistema nepasiūlo alternatyvos tai prekei (pvz. pigesnis kito gamintojo variantas)}
					\end{itemize}
		\subsubsection{Būsimasis scenarijus}
			\subsubsubsection{Pradedančio aludario būsimasis scenarijus}
				\begin{enumerate}
					\item{Petras nueina į aludarystės internetinę parduotuvę ir bando gauti informaciją apie alaus gamybą namuose}
					\item{Pagrindiniame puslapyje vartotojas paspaudžia ,,PRADEDANTIESIAMS"}
					\item{Atsidaro langas, kuriame pateiktas alaus gaminimo instruktažas teksto forma ir video, demonstruojantis visus reikiamus žingsnius}
					\item{Išvardinti ingridientai yra iš karto pateikti su galimybe įsidėti į krepšelį pačiam instruktaže ir nusipirkti juos}
					\item{Petras iš karto įsideda visas reikiamas prekes į krepšelį ir paspaudžia krepšelio piktogramą, kuri atidaro formą, kurioje gali apžiūrėti ir krepšelį, ir užbaigti savo pirkimą}
					\item{Petras dar nė karto nebuvo prisijungęs, todėl turi suvesti papildomus duomenis apie save}
					\item{Petrui pasiūloma užsiregistruoti, tačiau registracija yra neprivaloma}
					\item{Petras nurodo savo apmokėjimo ir kontaktų duomenis, užsakymas yra išsiunčiamas parduotuvės savininkui.}
				\end{enumerate}
			\subsubsubsection{Patyrusio aludario būsimasis scenarijus}
				\begin{enumerate}
					\item{Matas nueina į aludarystės internetinę parduotuvę ir bando susirasti reikiamos įrangos ir ingridientų alaus gamybai.}
					\item{Pradiniame puslapyje pirkėjas mato visų turimų prekių kategorijas (įranga, apyniai, salyklas ir t.t.).}
					\item{Vartotojui pasirinkus kategoriją jam parodomas prekių grupės sub-kategorijos.}
					\item{Matas gali tęsti prekės paiešką paspausdamas ant sub-kategorijos piktogramos ir taip išsirinkti norimą prekę arba pasinaudoti prekių paieška, kurioje Matas gali naudotis įvairiais  pasirinkimais ir filtrais, norėdamas rasti konkrečią prekę. Vartotojui taip pat suteikiama galimybė vykdyti paiešką į paieškos laukelį įvedus raktinius žodžius.}
					\item{Nepavykus surasti prekės paieškos laukelio pagalba Matui yra pasiūlomos panašios prekės pagal laukelio reikšmę.}
					\item{Prekių sąraše prie kiekvienos prekės piktogramos pateikiama trumpa prekės charakteristika. }
					\item{Matas prekes gali išrūšiuoti pagal - kainą, gamintoją, pavadinimą, populiarumą, įvertinimą.}
					\item{Pasirinkus konkrečią prekę parodomas detalus prekės aprašymas, taip pat pateikiamos panašios prekės.}
				\end{enumerate}
	\subsection{Panaudojamumo siekiai ir matai}
		\begin{itemize}
			\item{Naujas vartotojas informaciją apie gamybos procesą galės surasti ne daugiau kaip per 2 veiksmus}
			\item{Naujas vatotojas prekes iš instruktažo galės įsigyti ne daugiau kaip per 5 veiksmus}
			\item{Naujas vartotojas prekes galės nusipirkti neužsiregistravęs į parduotuvę}
			\item{Patyręs vartotojas konkrečią prekę galės surasti ne daugiau kaip per 5 veiksmus}
			\item{Patyrusiam varototojui bus pateikta detali paieškos sistema}
			\item{Vartotojui bus suteikta galimybė rušiuoti prekes pagal aktualius kriterijus}
			\item{Patyrusiam varotojui prekių saraše bus pateikta trumpa kiekvienos prekės charakteristika}
			\item{Kiekviena prekė turės detalų aprašymą ir naudojimosi instrukciją, jeigu ji yra reikalinga}
			\item{Vartotojui bus suteikta galimybė susisiekti su pardavėju dėl konsultacijos apie prekes}
		\end{itemize}

\section{Pardavėjo poreikiai}
	\subsection{Naudotojų charakteristikos}
		\subsubsection{Informacinių technologijų priemonės}
			Moderni interneto naršyklė, nešiojami kompiuteriai, stacionarūs kompiuteriai.
		\subsubsection{Motyvacija ir galimybės tobulėti}
			Pardavėjas bus gerai susipažinęs su parduotuvės sistema. Pardavėjęs gali turėti regos sutrikimų (silpna rega).
		\subsubsection{Veiklų kontekstai}
			Pardavėjui reikia sukurti ir pristatyti detalias instrukcijas kaip naudotis sistema, nes po sistemos perdavimo komunikacija su sistemos kūrėjais bus limituotas. Svarbu paruošti instrukciją, kuria galėtų sekti nauji pardavėjai.
		\subsubsection{Naudotojų tipai}
			Pardavėjas - kasdien naudojantis sistemą užsakymams apdoroti. Būtų galima išskirti dvi grupes, naujas pardavėjas ir pardavėjas su patirtimi, tačiau įsisavinti sistemą iš pardavėjo pusės neturėtų trukti ilgai.
	\subsection{Kompiuterizuojamų veiklų analizė}
		\subsubsection{Konceptinis scenarijus}
			Jonas yra pardavėjas, siekiantis kontroliuoti ir įgyvendinti užsakymus, kuriuos pateikia pirkėjai. Jonas atsidaro pardavėjo sistemą ir mato sąrašą su visais užsakymais. Užsakyme nurodyta, kokios prekės buvo užsakytos, apmokėjimo būdas, kaina ir užsakymo būsena(Apmokėta, neapmokėta, paruošta atsiėmimui, atsiimta). Jonas atsidaro užsakymų sąrašą, pasirenka dar neparuoštą atsiėmimui užsakymą, surenka prekes iš parduotuvės ir paruošia atsiėmimui. Pristačius arba vartotojui pačiam atsiėmus prekes, pardavėjas atnaujina užsakymo būseną į ,,Atsiimta". 
		\subsubsection{Veiklų charakteristikos}
			\subsubsubsection{Veiklos dažnis}
				
		\subsubsection{Problemos ir tobulėjjimo galimybės}
		\subsubsection{Būsimasis scenarijus}
	\subsection{Panaudojamumo siekiai ir matai}

\section{Įkvepiančios esamų interfeisų idėjos}
	\subsection{Paieškos interfeisas}
		\begin{figure}[h]
			\centering
			\includegraphics[width=6cm,height=18cm,keepaspectratio]{IkvepiantisInterfeisas1.png}
			\caption{ Paieška}
		\end{figure}

			Šis interfeiso pavyzdys(1 pav.) yra paimtas iš www.1a.lt internetinės parduotuvės prekių paieškos. 
			Vartotojas gali pasirinkti paieškos kriterijus paspausdamas ,,check box" mygtukus, kriterijai gali būti sumažinami siekiant lengviau peržiūrėti visų kriterijų sąrašą. 
			Šią interfeiso idėją būtų galima perpanaudoti alaus įrangos ir ingridientų paieškai mūsų sistemoje.
	\pagebreak
	\subsection{Prekės charakteristikos interfeisas}
		
	
		  \begin{figure}[h]
			\centering
			\includegraphics[width=6cm,height=18cm,keepaspectratio]{IkvepiantisInterfeisas2.png}
			\caption{ Prekės charakteristikos}
		\end{figure}

			Dar vienas interfeiso pavyzdys iš www.1a.lt. 2 pav.
			 paveikslėlyje parodyta, kaip atrodo trumpas prekės aprašymas visų prekių sąraše.
			 Šią idėją galima perpanaudoti rodant prekių informaciją aludarystės parduotuvėje.
			 Igyvendinus šį interfeisą vartotojui nereiktų atsidarinėti kiekvienos prekės puslapio, norint sužinoti bazinę informaciją.

	\pagebreak
	\subsection{Siūlomų panašių prekių interfeisas }
		\begin{figure}[h]
			\centering
			\includegraphics[width=6cm,height=18cm,keepaspectratio]{IkvepiantisInterfeisas3.png}
			\caption{ Prekės charakteristikos}
		\end{figure}

			Šiame pavyzdyje(3 pav.) pateikiamas www.amazon.com interfeiso sprendimas, rodantis prekes, kurias pirko kiti vartotojai, kurie taip pat pirko prekę, kurią vartotojas apžiūri(Customers who bought this item also bought). 
			Šią idėją galima perpanaudoti aludarystės sistemoje parodant vartotojui panašias ir/ar pigesnias prekes į tą prekę, kurią vartotojas dabar apžiūri.

	\pagebreak
	\subsection{Interfeisas naujiems vartotojams}
		\begin{figure}[h]
			\centering
			\includegraphics[width=6cm,height=18cm,keepaspectratio]{IkvepiantisInterfeisas4.png}
			\caption{ Prekės charakteristikos}
		\end{figure}
	
			Paimtas interfeiso pavyzdys(4 pav.) iš internetinės parduotuvės www.varle.lt. 
			Žmonės labiau linkę pirkti prekes su nuolaida negu be jos, todėl pirmas dalykas, kurį žmogus pastebi yra didelis žalias mygtukas ,,IŠPARDAVIMAS". 
			Šią dėmesio patraukimo idėja būtų galima pritaikyti atkreipiant naujų aludarių dėmesį nuo ko pradėti. 
			Aludarystės parduotuvės viršuje padaryti didelį ryškų mygtuką ,,PRADEDANTIESIAMS" ar kažką panašaus atkreipiant naujų aludarių dėmesį.

	
	\subsection{Atsiskaitymo interfeisas}
		\begin{figure}[h]
			\centering
			\includegraphics[width=6cm,height=18cm,keepaspectratio]{IkvepiantisInterfeisas5.png}
			\caption{ Prekės charakteristikos}
		\end{figure}

			Šiame pavyzdyje (5 Pav.) matome www.varle.lt interfeiso sprendimą atsiskaitant už prekes. 
			Už prekes galima atsiskaityti nesiregistruojant, prie parduotuvės galima prisijungti su Facebook arba Google paskyra, kuri po prisijungimo užpildys dalį duomenų, reikalingų prekės užsakymui, ir pati atsiskaitymo forma yra trumpa, aiški, visa informacija pateikiama viename lange. 
			Įgyvendinus šio interfeiso idėją aludarystės parduotuvėje vartotojas galėtų atsiskaityti už prekes greičiau.


\section{Terminų žodynėlis}
\section{Priedai}

\end{document}
