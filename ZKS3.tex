\documentclass[oneside]{VUMIFPSkursinis}
\usepackage{algorithmicx}
\usepackage{algorithm}
\usepackage{algpseudocode}
\usepackage{amsfonts}
\usepackage{float}
\usepackage{amsmath}
\usepackage{bm}
\usepackage{caption}
\usepackage{color}
\usepackage{float}
\usepackage{graphicx}
\usepackage{listings}
\usepackage{subfig}
\usepackage{ltablex}
\usepackage{longtable}
\usepackage{wrapfig}
\usepackage{subfig}
\usepackage{pbox}
\renewcommand{\labelenumii}{\theenumii}
\renewcommand{\theenumii}{\theenumi.\arabic{enumii}.}
\renewcommand{\labelenumiii}{\theenumiii}
\renewcommand{\theenumiii}{\theenumii\arabic{enumiii}.}
\newcolumntype{P}[1]{>{\centering\arraybackslash}p{#1}}
\usepackage[%
	colorlinks=true,
	linkcolor=black
]{hyperref}
\university{Vilniaus universitetas}
\faculty{Matematikos ir informatikos fakultetas}
\department{Programų sistemų katedra}
\papertype{Žmogaus ir kompiuterio sąsaja laboratorinis darbas III}
\title{Alaus gamybos įrangos ir ingredientų pirkimo sistema}
\titleineng{Aludarystės internetinė parduotuvė}
\status{3 kurso studentai}
\author{Greta Pyrantaitė}
\secondauthor{Matas Savickis}
\thirdauthor{Andrius Bentkus}

\supervisor{Kristina Lapin, Doc., Dr.}
\date{Vilnius – \the\year}

\bibliography{bibliografija}

\begin{document}
\maketitle

\sectionnonum{Anotacija}
Šio darbo tikslas išanalizuoti antramia laboratoriniame darbe sukurtus vartotojo sąsajos maketus. Analizė bus atlikta pagal euristikos ir pažintinės peržvalgos metodika. Kiekvienas grupės narys išanalizuos kitų dviejų grupės narių maketus. Išvadose bus pateikiama detaliojo prototipo projektiniai sprendimai.

\begin{itemize}
	\item{Greta Pyrantaitė - greta.pyrantaite@gmail.com}
	\item{Matas Savickis - savickis.matas@gmail.com}
	\item{Andrius Bentkus - andrius.bentkus@gmail.com}
\end{itemize}

\tableofcontents

\sectionnonum{Įvadas}
	\subsectionnonum{Programų sistemos ilgasis pavadinimas}
		Alaus gamybos įrangos ir ingredientų pirkimo sistema.
	\subsectionnonum{Programų sistemos trumpasis pavadinimas}
		Aludarystės internetinė parduotuvė
	\subsectionnonum{Dalykinė sritis}
		Elektroninė parduotuvė skirta aludariams.
	\subsectionnonum{Probleminė sritis}
		Sistema turi suteikt galimybę nusipirkti alaus gamybai reikalingus ingredientus ir įrangą bei gauti visą reikalingą informaciją apie ingredientų ir įrangos specifikacijas.
		Sistema taip pat turi vartotojui pateikti informaciją apie alaus gamybos procesą panaudojant nusipirktus ingredientus ir įrangą.
	\subsectionnonum{Naudotojai}
		\begin{itemize}
			\item{Pirkėjas(aludaris) - pirkėjas turi galėti užsisakyti alaus gamybos ingridientus ir įrangą bei gauti reikalingą specifikaciją norint naudotis preke.
				Pirkėjui turi užtekti mokyklinio informatikos kurso žinių ir bendro supratimo, kaip naviguotis internetinėse svetainėse.}
			\item{Pardavėjas - pardavėjui sistema turi suteikti informaciją apie užsakymus, jų apmokėjimus ir panašią svarbią informaciją.
				Sistema pardavėjui taip pat turi suteikti galimybę pridėti arba išimti prekes iš internetinės parduotuvės.
				Pardavėjui taip pat turi užtekti mokyklinio lygio informatikos žinių ir bendrų žinių naviguotis interneto svetainėse.}
		\end{itemize}
	\subsectionnonum{Naudoti dokumentai}
		Kristina Lapin - Žmogaus ir kompiuterio sąveikos paskaitų skaidrės, laboratorinių darbų aprašymai.
\iffalse XXXXXXXXXXXXXXXXXXXXXXXXXXXXXXXXXXXXXXXXXXXXXXXXXXXXXXXXXXXXXXXXXXXXXXXXXXXXXXXXXXXXXXXXXXXXXXXXXXXXXXXXXXXXXXXXXXXXXXXXXXXXXXXXXXXXXXX \fi
\section{Mato Savickio maketo vertinimas}
	\subsection{Euristinis tikrinimas}
Vertinimo autorė Greta Pyrantaitė
\begin{center}
    \begin{tabular}{ |p{3cm}| p{3cm} | p{11cm} | }
    \hline
    	Euristika &Defekto sunkumas & Komentaras \\ \hline 
	Būsenos matomumas & Vidutinis & Vartotojui nėra pateikiama kelio kuriuo jis norėjo navigacijos pavidalu, pavyzdžiui duonos trupiniais. Įvertinau vidutiniu sunkumu, nes pridėjus daugiau funkcijų vartotojas pasimes lengviau. \\ \hline
	Būsenos matomumas & Lengvas & Vartotojui nėra pateikta kuriame ekrane jis dabar yra. Kaikurie ekranai turi antraštes tačiau ne visi. \\ \hline
	Naudotojo valdomas dialogas & Lengvas & Iš atsiskaitymo lango vartotojui nesuteikta galimybė grįšti į prekių sąraša paspaudus mygtuką pagrindiniame lange. Vartotojas gali paspausti krepšelio ikoną esančia viršuje, tačiau ne naujas pirkėjas gali jos nepastebėti, suirzti ir nebenusipirkti prekių. \\ \hline
	Darna ir standartai & Lengvas & Prekių atsiskaitymo lango apačioje yra mygtukas baigti pirkimą kuris įvykdo prekių pirkimą. Vartotojas gali būti suklaidintas ir manyti, kad mygtukas atšaukia visą pirkimą ir nenuperka prekių. \\ \hline 
	Klaidų prevencija & Lengvas & Atsiskaitymo ekrane paspaudus mygtuką ,,Baigti pirkimą" pasirodo pranešimo langas ,,Ar tikrai norite baigti pirkimą?" kurį vartotojas gali interpretuoti kaip pirkimo panaikinimą ir nenoromis nusipirkti prekes. Po nusipirkimo nėra pateikta funkcijų užsakymo atšaukimui. \\ \hline 
	Naudojimo lankstumas ir našumas & Lengvas & Tinklalapyje nėra įgyvendinta greitųjų santrumų ir pasirinkimų. Defektas lengvas nes svetaine naudojamasi trumpai ir salyginai retai \\ \hline
	Pagalba ir dokumentacija & Vidutinis & Tinklalapis neturi numatytos pagalbos ar dokumentacijos funkcijos, sudėtingesnius konceptus vartotojas turi išsiaiškinti pats. Lygis vidutinis, nes apmokymų trūkumas, gal pabloginti patyrusio vartotojo arba vartotojo norinčio tapti patyrusiu patirtį \\ \hline
	
   \hline
    \end{tabular}
\end{center}
\pagebreak
	\subsection{Pažintinė peržvalga}

\begin{center}
    \begin{tabular}{ |p{4cm}| p{6cm} | p{7cm} | }
    \hline
    	Užduoties žingsniai & Ar aiškiai matoma, ką daryti? & Ar suprantamas atsakas \\ \hline 
	Pagrindiniame ekrane surasti nuorodą į pasigaminimo instrukciją & Pagrindiniame lange dviejuose matomiausiose vietose pateikti mygtukai vedantys į pasigaminimo.
	Mygtukai lengvai pastebimi & Paspaudus ant mygtuko vartotojas nukeliamas į naują langą. 
	Vartotojui aiškiai suprantama, kad nuėjo ten kur reikiai nes rodoma puslapio antraštė. Būtų galima pridėti duonos trupinių navigacijos juostą siekias dar aiškiau parodyti kur vartotojas yra. \\ \hline
	Instrukcijos gavimas & Gamybos instrucijų ekrane pateikiamas tiek vaizo instruktažas tiek ir rašytinė instrukciją kaip vykdyti alaus galybą. Vartotojas įsisavina informacija efektyviai & Vartotojui informacija pateikta suprantamai \\ \hline
 	Prekių iš instruktažo pirkimas & Gaminimo instrukcijos lango kairėje pusėje vartotojas mato ,,vėžimėlio" ikoną ir supranta, kad prekes pristatymas per instruktaža gali įsidėti į krepšelį ir nusipirkti & Renkantis po vieną prekę vartotojui neparodomas pranešimas apie įdėtą prekę, reiktų padaryti informacinį pranešimą. Tačiau įdėjus visas prekes vartotojas nukreipiamas į krepšelį kur mato visas prekes \\ \hline
	Prekių panaikinimas iš krepšelio & Prekių krepšelio lange vartotojas mato visas įsidėtas prekes ir šalia jų ,,X" mygtuką kuris indikuoja, kad prekes galima ištrinti. & Paspaudus ,,X" vartotojui parodomas informacinis pranešimas klausiantis ar tikrai nori atlikti veiksmą ir pasirinkus taip prekė dingsta iš krepšelio \\ \hline
	Ėjimas į atsiskaitymo langą & Prekių saraše yra vienas mygtukas ,,Atsiskaityti" kurį paspaudęs vartotojas nukeliamas į atsiskaitymo langą & Paspaudus mygtuką vartotojas persikelia į atsiskaitymo langą. Jis tai supranta iš to, kad lango antraštė rašo ,,Atsiskaitymas" \\ \hline
	Atsiskaitymas už prekes & Atsiskaitymo ekrane yra trys išskreidžiami elementai susiję su atsiskaitymų. Jeigu vartotojas suklysta vienoje vietoje(pvz neteisingai nurodęs pristatymo informacija) visados gali ją pakeisti atsidaręs pasirinkimą & Vartotojui gali pasirodyti neaišku, kad atidarant vieną tsiskaitymo elementą užsidaro kitas, reiktų leistiatidaryti visus. Pabaigęs pildyti atsiskaitymo informacija vartotojas spaudžia ,,baigti užsakymą" ir jam yra parodomas langas su informacija, kad užsakymas atliktas teisingai \\ \hline
	Prekių sub-kategorijų rodymas neperkraunant lango & Pagrindiniame programos lange vartotojui paspaudus vieną iš sąrašo elementų kairėje išsiskleidžia sub-kategorijų sąrašas & Išsisikleidžiant sub-kategorijom vartotojui gali būti neaišku, kodėl jam neperkrovė puslapio ir neparodė prekių sąrašo. Reiktų išskirstyti mygtuką į rodyklę kuri išskeidžia sub-kategorijas ir paprasta mygtuką rodantis visas to tipo prekes \\ \hline

   \hline
    \end{tabular}
\end{center}
	\subsection{Apibendrinimas}
\iffalse XXXXXXXXXXXXXXXXXXXXXXXXXXXXXXXXXXXXXXXXXXXXXXXXXXXXXXXXXXXXXXXXXXXXXXXXXXXXXXXXXXXXXXXXXXXXXXXXXXXXXXXXXXXXXXXXXXXXXXXXXXXXXXXXXXXXXXX \fi
\section{Gretos Pyrantaitės maketo vertinimas}
	\subsection{Euristinis tikrinimas}
Mato Savickio vertinimas
\begin{center}
    \begin{tabular}{ |p{3cm}| p{3cm} | p{11cm} | }
    \hline
	Euristika &Defekto sunkumas &Komentaras \\ \hline
	Būsenos matomumas & Lengas & Pagrindiniame ekrane niekur neindentifikuota kokiame ekrame yra vartotojas. 
						Vartotojas turi suprasti iš patirties. 
						Defekto sunkumas lengvas nes yra tik dvi pagrindinės būsenos ir jų funkcionalumas ženkliai skiriasi. \\ \hline
	Būsenos matomumas & Sunkus & Vartotojui pakeitus užsakymo būseną vartotojas niekaip neinformuojamas, kad būsena buvo pakeista sėkmingai.
						Sunkumo lygis aukštas, nes vartotojui norint įsitikinti, kad veiklą atliko teisingai kiekvieną kartą reikės saraše ieškoti užsakymo elemento kurį pakeitė, atsidaryti jį ir tik tuo metu pasižiūrėti ar būsena buvo pakeista teisingai. \\ \hline
	Būsenos matomumas & Lengvas & Prekės informacijos redagavimo lange redaguojant prekės informaciją ir ją išsaugant vartotojas niekaip neinformuojamas, kad informacija pakeista sėkmingai.
						Sunkumo lygis žemas, nes vartotojas iškarto nukeliamas į prekės aprašymą todėl gali pats pasižiūrėti ar viskas gerai, tačiau mažų pakeitimų reikės ieškoti ilgiau. \\ \hline
	Naudotojo valdomas dialogas & Vidutinis & Vartotojo sąsajoje nėra pateikta kelio kurį nuėjo vartotojas(duonos trupiniai). 
						Sunkumas vidutinis, nes varotojas netyčią paspaudęs jam nežinomą mygtuką atsiduria jam nežinomoje vietoje ir nežino kaip iki ten atėjo. \\ \hline
	Darna ir standartai & Vidutinis & Užsakymo informacijos lange vartotojui pateikti du mygtukai ,,Pakeisti būseną" ir ,,Atšaukti".
						Mygtumas atšaukti uždaro informacijos langą, tačiau šis mygtukas gali būti suprastas kaip užsakymo atšaukimas arba panaikinimas iš sąrašo. Sunkumas vidutinis, nes vartotojui pirmus kelis kartus mygtumo paskirtis gali pasirodyti dviprasmiška. \\ \hline

   \hline
    \end{tabular}
\end{center}
\begin{center}
    \begin{tabular}{ |p{3cm}| p{3cm} | p{11cm} | }
	\hline
	Klaidų prevencija & Sunkus & Prekės informacijos redagavimo lange yra mygtukas ,,Išimti prekę". 
					Vartotojui paspaudus jį pasirodo pranešimas klausiantis ,,Ar tikrai norite ištrinti prekę". 
					Paspaudus jį prekė yra pašalinama iš sąrašo.
					Nėra pateikta jokios ,,šiukšledėžės" arba ,,undo" galimybės, todėl vartotojui per klaidą patvirtinus prekės pašalinimą visa svarbi prekės informacija yra pašalinama nesugrįštamai. \\ \hline
	Naudojimo lankstumas ir našumas & Lengvas & Vartotojui nėra suteikiama galimybė naudotis klavišų santrumpom, pelės santumpom, navigacijos šuoliais ar panašiais metodais palengvinančiais darbą.
								Šis pažeidimas įvertintas kaip lengvas, nes programoje nėra daug funkcionalumo todėl tumpiniai daug laiko nesutaupytų. \\ \hline
	Pagalba ir dokumentacija & Lengvas & Sistema vartotojui nesuteikia jokios dokumentacijos, apmokymų ar aprašymų kaip naudotis sistema. 
							Įvertinta kaip lengvas defektas nes funkcionalumo yra nedaug ir jis išmokstamas pačiam bandant naudotis sistema. \\ \hline
   \hline
    \end{tabular}
\end{center}
	\subsection{Pažintinė peržvalga}
Mato Savickio peržvalga
\begin{center}
    \begin{tabular}{ |p{4cm}| p{6cm} | p{7cm} | }
    \hline
    	Užduoties žingsniai & Ar aiškiai matoma, ką daryti? & Ar suprantamas atsakas \\ \hline 
	Įsijungti užsakymų sąrašą & Vartotojas aiškiai mato kairėje pusėje esančia navigacijos juostą.
					Vartotojas paspaužia ,,Užsakymai" & Vartotojui pateikiamas visų užsakymų prekių sąrašas suskirstytas į grupes \\ \hline
	Pasirinkti užsakymą & Kiekvienas užsakymo elementas saraše yra išskirtas į atskirą mygtuko formos laukelį ir vartotojui aišku, kad reikia spausti ant norimo užsakymo & Šalia užsakymo elemento atsiranda langas su detalia užsakymo informacija vartotojui aišku, kad informacija yra apie elementą kurį paspaudė \\ \hline
	Vartotojas keičia užsakymo būseną & Detalios informacijos lange yra matomas mygtukas ,,Pakeisti būseną", vartotojas supranta, kad reikia paspausti ant jo & Neuždarant informacijos lango atsidaro kitas langas, kuriame vartotojas iš ,,drop down" sąrašo pasirenka būsena  Pakeitus būseną ir paspaudus ,,išsaugoti" vartotojas iškarto nukeliamas į užsakymų langą.
Jam gali būti neaišku ar būsena ištikrūjų buvo pakeista, nes neparodomas joks pranešimas apie sėkmingai pakeista būseną. \\ \hline
	
   \hline
    \end{tabular}
\end{center}

\begin{center}
    \begin{tabular}{ |p{4cm}| p{6cm} | p{7cm} | }
    \hline
	Peržiūrėti prekės informaciją & Prekių saraše prie kiekvienos prekės yra mygtukas ,,peržiūrėti".
	Vartotojui aišku, kad norint peržiūrėti individualios prekės informaciją reikia spausti šį mygtuką & Atsidaro prekės informacijos langas, todėl vartotojui aišku, kad jo įvesti buvo korektiška \\ \hline
	Atsidaryti redagavimo langą & Prekės informacijos lange vartotojas mato mygtuką ,,Redaguoti".
	Jam yra aišku, kad mygtukas įjungs prekės redagavimą. Galimas patobulinimas yra leisti redaguoti iškarto, be papildomo lango atidarymo taip sumažinant paspaudimų skaičių. 
	& Atsidaro naujas langas kuriame matoma, kad galima redaguoti prekę, gali kilti šiek tiek neaiškumo, nes langas labai panašus į praeitą informacijos langą, todėl juos reikų sujungti \\ \hline
	Išsaugoti pakeitimus & Prekės redagavimo lange vartotojas  mato mygtuką ,,išsaugoti pakeitimus", paspaudus jį išsaugoma paredaguota prekės informacija, tačiau vartotojas neinformuojamas apie sėkmingą pakeitimą & Vartotojui gali būti neaišku ar jo pakeitimas buvo išsaugotas ar langas tiesiog užsidarė, todėl reiktų parodyti pranešimą pavirtinanti, kad prekės informacija pasikeitė \\ \hline
	Išimti prekę & Prekės informacijos lange vartotojas mato mygtuką ,,Išimti prekę". Vartotojui paspaudus mygtuką jam parodomas informacinis langas kuris prašo vartotojo patvirtinti šį veiksmą & Patvirtinus išėmimo veiksmą langas tiesiog išsijungia, be patvirtinimo kad veiksmas įvyko teisingai. Reiktų padaryti pranešimo langą patvirtinanti, kad operacija įvyko korektiškai \\ \hline
	Vartotojas ištrina užsakymą & Užsakymo informacijos lange vartotojas mato mygtuką ,,Atšaukti". Vartotojas gali šį lygtuką interpretuoti kaip ,,Atšauti užsakymą" tačiau mygtuko funkcija yra išeiti iš informacijos lango.&
	Reikia pridėti papildomą mygtumą ,,Atšaukti užsakymą" ir pakeisti esamą mygtumą į ,,Eiti atgal" \\ \hline
	Įsijungti prekių sąrašą & Navigacijos juostoje vartotojas mato mygtumą ,,Mano prekės", jam aišku, kad paspaudus ant jo bus parodytas prekių sąrašas & Paspaudus mygtuką vartotojui parodomas prekių sąrašas \\ \hline
   \hline
    \end{tabular}
\end{center}
\pagebreak
	\subsection{Apibendrinimas}
\iffalse XXXXXXXXXXXXXXXXXXXXXXXXXXXXXXXXXXXXXXXXXXXXXXXXXXXXXXXXXXXXXXXXXXXXXXXXXXXXXXXXXXXXXXXXXXXXXXXXXXXXXXXXXXXXXXXXXXXXXXXXXXXXXXXXXXXXXXX \fi

\section{Andriaus Bentkaus maketo vertinimas}
	\subsection{Euristinis tikrinimas}
Mato Savickio vertinimas
\begin{center}
    \begin{tabular}{ |p{3cm}| p{3cm} | p{11cm} | }
    \hline
   	 Euristika &Defekto sunkumas &Komentaras \\ \hline 
	Būsenos matomumas & Vidutinis & Vartotojo sąsaja neturi nueito kelio navigacinės juostos(duonos trupiniai). Sunkumas vidutinis, nes galutinėje sistemoje bus daug prekių kategotijų ir pasirinkimų todėl vartotojui bus sunku susikaudyti iš kur jis atėjo. \\ \hline
	Naudotojo valdomas dialogas & Vidutinis & Nerodomas vartotojo nueitas kelias. Vidutinė, nes įvedus daug funkcionalumų bus sunku susigaudyti kur esi. \\ \hline
	Naudojimo lankstumas & Lengvas & Vartotojo sąsaja neturi klavišų santrumpų, pelės santrumpų ar kitų panašių funkcionalumų pagritinančių darbą.
							Sunkumas lengvas dėl to, kad parduotuvedažniausiai naudojamasi kartą į mėnesį todėl santrumpos labai daug laiko nesutaupytų. \\ \hline
	Estetika ir minimalizmas & Lengvas & Pagrindiniame programos lange du kartus pakartotos kategorijos, šoninėje navigavijos juostojo ir paskui apačioje prie ikonų. Sunkumas lengvas, nes iš vienos pusės vartotojo dėmesį trigdo daugybė piktogramų iš kitos pusės tai padeda lengviau surasti norimą kategoriją \\ \hline
	Būsenos matomumas & Vidutinis & ,,Kaip pasigaminti" lange vartotojui paspaudus ,,Įsidėti visas prekes" ir tada paspaudus ,,Taip" nepateikiamas joks pranešimas, kad prekės buvo sėkmingai  pirdėtos į krepšelį. Vidutinis, nes vartotojas gali susierzinti pirmą kartą pirkdamas prekes, nerasti kur jos atsirado ir palikti tinklalapį nenusipirkęs prekių. \\ \hline
	Pagalba ir dokumentacija & Vidutinis & Informacija yra suteikiama tik pradedantiems aludariams.
							Tinklalapyje nėra pateiktos informacijos padedančios lengviau įtraukti entiziastus ir giliau supažindinti juos su aludarystės tinklalapio veikimu ir sąvokomis. Vidutinis, nes yra galimybė prarasti pirkėjus pasirengusius sumokėti daug pinigų už brangią įrangą tačiau neradusiems informacijos apie ją. \\ \hline

	
	
   \hline
    \end{tabular}
\end{center}
\pagebreak

Vertinimo autorė Greta Pyrantaitė
\begin{center}
    \begin{tabular}{ |p{3cm}| p{3cm} | p{11cm} | }
    \hline
	Euristika &Defekto sunkumas & Komentaras \\ \hline
	Vartotojo valdomas dialogas & Vidutinis & Vartotojas neturi galimybės atpažinti kuriame puslapyje yra, nes yra navogacijos juostos padedančios orientuotis. Vartotojas vietą tinklapyje turi suprasti iš patirties. Įvertinau vidutiniu sunkumu, nes kai bus sukurta daugiau puslapiu naudotojas lengvai pasimes tarp jų \\ \hline
	Naudojimo lankstumas ir našumas & Lengvas & Tinklalapis neįgalina naudotojo naudotis trumposionis klavišų kombinacijomis taip pasunkindamas darbą patyrusiems naudotojams. Tačiau tinklalapiu nebus naudojamasi dažnai todėl ir trumposios kombinacijos nebus tokios efektyvios. \\ \hline
	Estetika ir minimalizmas & Lengvas & Pirmajame svetainės ekrane naudotojui parodoma daug ikonų ir jos blaško dėmesį, naudotojas gali pasimesti tarp jų. Sunkumas lengvas, nes vertotojui reikia kažkur pateikti visą informacija ir galbūt negalima supaprastinti tinklalapio. \\ \hline
	Būsenos matomumas & Lengvas & Naudotojui peržiūrėjus instruktažas ,,Kaip pasigaminti" ekrane ir spaudžiant pridėti prekes naudotojas neinformuojamas, kad prekės sėkmingai pridėtos. \\ \hline
	Pagalba ir dokumentacija & Lengvas & Naudotojui nesuteikiama dokumentcija ir pagalba. Lengvas nes vartotojas pats gali viską išmokti \\ \hline	 
   \hline
    \end{tabular}
\end{center}


	\subsection{Pažintinė peržvalga}
Mato Savickio peržvalga
\begin{center}
    \begin{tabular}{ |p{4cm}| p{6cm} | p{7cm} | }
    \hline
    	Užduoties žingsniai & Ar aiškiai matoma, ką daryti? & Ar suprantamas atsakas \\ \hline 
	Peržiūrėti konkrečios kategorijos prekes & Pagrindinio ekrano kairėje pusėje yra sąrašas su visomis prekių kategorijos aiškiai patomas vartotojo & Paspaudus ant vienos iš prekių ekrane pasirodo visas tos kategorijos prekių sąrašas, o navigacijos juostoje ,,drop down" principu parodomos sub-kategorijos, vartotojui aišku, ką jis pasirinko. \\ \hline
	Sub-kategorijos pasirinkimas & Vartotojas nori pasirinkti prekės subkategorija. Kairės pusės navigacijos juostoje parodomas nusileidęs ,,drop down" sąrašas rodantis sub-kategorijas ir vartotojas gali pasirinkti vieną iš jų & Vartotojui pasirinkus subkategoriją nėra visiškai aišku ar veiksmas įvyko, nes nieko neparodoma puslapio antraštė. Prekės gali būti panašios ir klaidinti vartotoja, kad jam nepavyko pasirinkti subkategorijos. Siūloma padaryti puslapio antraštę, kad vartotojui būtų aišku, kur jis dabar yra. \\ \hline
	Prekės pasirinkimas iš sąrašo & Vartotojas mato visą prekių sarašą, kiekviena prekė turi trumpą savo aprašymą. 
	Prekes galima rušiuoti. Vartotojui aiškiai pateikta informacija & Pasirinkus norimą prekę atsidaro naujas langas pateikiantis detalų prekės aprašymą. 
Vartotojui aišku, kad pasikeitė tinklapio būsena. \\ \hline
	Alaus gaminimo instrukcijos radimas & Vartotojas pagrindinio puslapio viršuje esančioje navigacijos juostoje aiškiai mato mygtuką ,,Kaip pasigalinti?" & Vartotojui paspaudus mygtuką atsidaro langas su instrukcijomis. Vartotojui aišku kokiame ekrane jis atsidūrė, nes puslapio viršuje užrašytas puslapio pavadinimas \\ \hline
	Prekių pirkimas po instrukcijų išklausymo & Pasigaminimo instrukcijos ekrano kairėje pusėje vartotojas mygtukus ,, Įdėti prekę" ir ,,Įdėti visas prekes" & Vartotojui gali būti neaišku apie kurias individualias prekes kalbama, nes pateiktas tik vienas mygtukas su prekės pavadinimu be aprašymo.
	Paspaudus ,,Įdėti visas prekes" ir patvirtinus veiksmą neparodomas pranešimas, kad prekės buvo pridėtos sėkmingai. Instrukcijos lange reiktų detalesnio prekių aprašymo, o patrivtinus prekes reiktų pranešimo patvirtinančio būsenos pasikeitimą \\ \hline
	
   \hline
    \end{tabular}
\end{center}

Pažintinės peržvalgos autorė Greta Pyrantaitė

\begin{center}
    \begin{tabular}{ |p{4cm}| p{6cm} | p{7cm} | }
    \hline
	Užduoties žingsniai & Ar aiškiai matoma, ką daryti? & Ar suprantamas atsakas \\ \hline 
	Surasti prekės kategoriją & Kairioje pusėje esančioje navigacioojos juostoje naudotojui pateikta visų prekių kategorijos ir jis gali lengvai rasti norimą kategoriją, taip pat kategorijos pateiktos puslapio apačioje. Vartotojas pasirenka kategorija &  Atsidaro langas su visomis prekėmis iš tos kategorijos. Vartotojui aišku kur jis pateko, bet reiktų ,,duonos trupinių" papildomam aiškumui \\ \hline
	Prekės išvestinės kategorijos pasirinkimas & Navigacijos juostoje paspaudus ant vienos iš prekių išsiskleidžia išvestiniu kategorijų sąrašas ,,hamburgerio" stiliumi, vartotojui aišku, kad išvestinės kategorijos pasirodė iš pagrindinės prekės. Vartotojas spaudžia ant išvestinės kategorijos ir prekių sąrašas atsinaujina & Vartotojui gali būti nevisiškai aišku, ar buvo pasirinkta išvestinė kategorija, nes nėra jokios puslapio antraštės ar navigacijos kelio. Reiktų pridėti nueito kelio juostą ir puslapio antraštę \\ \hline
	Prekės pasirinkimas & Vartotojas iš visų pateikų prekių ieško konkrečios prekės. Jam paiešką palengvina filtravimo galimybės ir tumpas kiekvienos prekės specifikacijos aprašymas. Vartotojas nesunkiai suranda norimą prekę ir paspaudžia ant jos & Paspaudus ant prekės atsidaro naujas langas su detalia prekės informacija. Vartotojui aišku, kad į jo paspaudimą tinklalapis suregavo \\ \hline
	Detali prekės informacija & Vartotojui paspaudus ant vienos iš prekų atsiveria naujas langas su detaliu prekės aprašymu. Vartotojui aišku, kad būsena pasikeitė taip kaip vis norėjo nes antraštėje yra prekės pavadinimas & Vartotojas supranta, kad jis yra kitame puslapyje ir is navogacijos juostos mato kur gali eiti toliau. \\ \hline
	Naudotojas ieško informacijos kaip pasigaminti alaus & Pagrindiniiame ekrane naudotojas ieško informacijos apie alaus gaminimą ir viršutinėje meniu juostoje aiškiai pastebi ,,Kaip pasigaminti" mygtuką. Vartotojui aišku, kad paspaudus jį jam bus suteikta informacija apie alaus gamyba  & Tinklapio atsakas vartotojui yra aiškus nes puslapis atidaro naują langą kurio antraštė yra ,,Pasigaminimo instrukcija" \\ \hline
	Prekių iš mokymų pirkimas & Vartotojas nori įsigyti prekes kurias matė instrukcijoje. Instrukcijų puslapio dešinėje pusėje yra pateiktas visas prekių sąrašas ir vartotojas gali  jas įsidėti į krepšelį. Prekių ikonos yra aiškiai matomos vartotojo. & Pridedant prekes į krepšelį vartotojas neinformuojamas, kad operacija įvykdyta sėkmingai, todėl varotojas gali sutrikti ir įsidėti prekes kelis kartus. Reikia sisteminio pranešimo išspėjančio vartotoja, kad prekės pridėtos sėkmingai \\ \hline
	
	
	
	
   \hline
    \end{tabular}
\end{center}

\pagebreak
	\subsection{Apibendrinimas}
\iffalse XXXXXXXXXXXXXXXXXXXXXXXXXXXXXXXXXXXXXXXXXXXXXXXXXXXXXXXXXXXXXXXXXXXXXXXXXXXXXXXXXXXXXXXXXXXXXXXXXXXXXXXXXXXXXXXXXXXXXXXXXXXXXXXXXXXXXXX \fi
\section{Išvados}


\end{document}
