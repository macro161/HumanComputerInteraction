\documentclass[oneside]{VUMIFPSkursinis}
\usepackage{algorithmicx}
\usepackage{algorithm}
\usepackage{algpseudocode}
\usepackage{amsfonts}
\usepackage{float}
\usepackage{amsmath}
\usepackage{bm}
\usepackage{caption}
\usepackage{color}
\usepackage{float}
\usepackage{graphicx}
\usepackage{listings}
\usepackage{subfig}
\usepackage{ltablex}
\usepackage{longtable}
\usepackage{wrapfig}
\usepackage{subfig}
\usepackage{pbox}
\renewcommand{\labelenumii}{\theenumii}
\renewcommand{\theenumii}{\theenumi.\arabic{enumii}.}
\renewcommand{\labelenumiii}{\theenumiii}
\renewcommand{\theenumiii}{\theenumii\arabic{enumiii}.}
\newcolumntype{P}[1]{>{\centering\arraybackslash}p{#1}}
\usepackage[%
	colorlinks=true,
	linkcolor=black
]{hyperref}
\university{Vilniaus universitetas}
\faculty{Matematikos ir informatikos fakultetas}
\department{Programų sistemų katedra}
\papertype{Žmogaus ir kompiuterio sąsaja laboratorinis darbas III}
\title{Alaus gamybos įrangos ir ingredientų pirkimo sistema}
\titleineng{Aludarystės internetinė parduotuvė}
\status{3 kurso studentai}
\author{Greta Pyrantaitė}
\secondauthor{Matas Savickis}
\thirdauthor{Andrius Bentkus}

\supervisor{Kristina Lapin, Doc., Dr.}
\date{Vilnius – \the\year}

\bibliography{bibliografija}

\begin{document}
\maketitle

\sectionnonum{Anotacija}
Šio darbo tikslas išanalizuoti antramia laboratoriniame darbe sukurtus vartotojo sąsajos maketus. Analizė bus atlikta pagal euristikos ir pažintinės peržvalgos metodika. Kiekvienas grupės narys išanalizuos kitų dviejų grupės narių maketus. Išvadose bus pateikiama detaliojo prototipo projektiniai sprendimai.

\begin{itemize}
	\item{Greta Pyrantaitė - greta.pyrantaite@gmail.com}
	\item{Matas Savickis - savickis.matas@gmail.com}
	\item{Andrius Bentkus - andrius.bentkus@gmail.com}
\end{itemize}

\tableofcontents

\sectionnonum{Įvadas}
	\subsectionnonum{Programų sistemos ilgasis pavadinimas}
		Alaus gamybos įrangos ir ingredientų pirkimo sistema.
	\subsectionnonum{Programų sistemos trumpasis pavadinimas}
		Aludarystės internetinė parduotuvė
	\subsectionnonum{Dalykinė sritis}
		Elektroninė parduotuvė skirta aludariams.
	\subsectionnonum{Probleminė sritis}
		Sistema turi suteikt galimybę nusipirkti alaus gamybai reikalingus ingredientus ir įrangą bei gauti visą reikalingą informaciją apie ingredientų ir įrangos specifikacijas.
		Sistema taip pat turi vartotojui pateikti informaciją apie alaus gamybos procesą panaudojant nusipirktus ingredientus ir įrangą.
	\subsectionnonum{Naudotojai}
		\begin{itemize}
			\item{Pirkėjas(aludaris) - pirkėjas turi galėti užsisakyti alaus gamybos ingridientus ir įrangą bei gauti reikalingą specifikaciją norint naudotis preke.
				Pirkėjui turi užtekti mokyklinio informatikos kurso žinių ir bendro supratimo, kaip naviguotis internetinėse svetainėse.}
			\item{Pardavėjas - pardavėjui sistema turi suteikti informaciją apie užsakymus, jų apmokėjimus ir panašią svarbią informaciją.
				Sistema pardavėjui taip pat turi suteikti galimybę pridėti arba išimti prekes iš internetinės parduotuvės.
				Pardavėjui taip pat turi užtekti mokyklinio lygio informatikos žinių ir bendrų žinių naviguotis interneto svetainėse.}
		\end{itemize}
	\subsectionnonum{Naudoti dokumentai}
		Kristina Lapin - Žmogaus ir kompiuterio sąveikos paskaitų skaidrės, laboratorinių darbų aprašymai.
\iffalse XXXXXXXXXXXXXXXXXXXXXXXXXXXXXXXXXXXXXXXXXXXXXXXXXXXXXXXXXXXXXXXXXXXXXXXXXXXXXXXXXXXXXXXXXXXXXXXXXXXXXXXXXXXXXXXXXXXXXXXXXXXXXXXXXXXXXXX \fi
\section{Mato Savickio maketo vertinimas}
	\subsection{Euristinis tikrinimas}
\begin{center}
    \begin{tabular}{ |p{3cm}| p{3cm} | p{11cm} | }
    \hline
    Euristika &Defekto sunkumas &Komentaras \\ \hline 
   \hline
    \end{tabular}
\end{center}
	\subsection{Pažintinė peržvalga}
\begin{center}
    \begin{tabular}{ |p{4cm}| p{6cm} | p{7cm} | }
    \hline
    Užduoties žingsniai & Ar aiškiai matoma, ką daryti? & Ar suprantamas atsakas \\ \hline 
   \hline
    \end{tabular}
\end{center}
	\subsection{Apibendrinimas}
\iffalse XXXXXXXXXXXXXXXXXXXXXXXXXXXXXXXXXXXXXXXXXXXXXXXXXXXXXXXXXXXXXXXXXXXXXXXXXXXXXXXXXXXXXXXXXXXXXXXXXXXXXXXXXXXXXXXXXXXXXXXXXXXXXXXXXXXXXXX \fi
\section{Gretos Pyrantaitės maketo vertinimas}
	\subsection{Euristinis tikrinimas}
Mato Savickio vertinimas
\begin{center}
    \begin{tabular}{ |p{3cm}| p{3cm} | p{11cm} | }
    \hline
	Euristika &Defekto sunkumas &Komentaras \\ \hline
	Būsenos matomumas & Lengas & Pagrindiniame ekrane niekur neindentifikuota kokiame ekrame yra vartotojas. 
						Vartotojas turi suprasti iš patirties. 
						Defekto sunkumas lengvas nes yra tik dvi pagrindinės būsenos ir jų funkcionalumas ženkliai skiriasi. \\ \hline
	Būsenos matomumas & Sunkus & Vartotojui pakeitus užsakymo būseną vartotojas niekaip neinformuojamas, kad būsena buvo pakeista sėkmingai.
						Sunkumo lygis aukštas, nes vartotojui norint įsitikinti, kad veiklą atliko teisingai kiekvieną kartą reikės saraše ieškoti užsakymo elemento kurį pakeitė, atsidaryti jį ir tik tuo metu pasižiūrėti ar būsena buvo pakeista teisingai. \\ \hline
	Būsenos matomumas & Lengvas & Prekės informacijos redagavimo lange redaguojant prekės informaciją ir ją išsaugant vartotojas niekaip neinformuojamas, kad informacija pakeista sėkmingai.
						Sunkumo lygis žemas, nes vartotojas iškarto nukeliamas į prekės aprašymą todėl gali pats pasižiūrėti ar viskas gerai, tačiau mažų pakeitimų reikės ieškoti ilgiau. \\ \hline
	Naudotojo valdomas dialogas & Vidutinis & Vartotojo sąsajoje nėra pateikta kelio kurį nuėjo vartotojas(duonos trupiniai). 
						Sunkumas vidutinis, nes varotojas netyčią paspaudęs jam nežinomą mygtuką atsiduria jam nežinomoje vietoje ir nežino kaip iki ten atėjo. \\ \hline
	Darna ir standartai & Vidutinis & Užsakymo informacijos lange vartotojui pateikti du mygtukai ,,Pakeisti būseną" ir ,,Atšaukti".
						Mygtumas atšaukti uždaro informacijos langą, tačiau šis mygtukas gali būti suprastas kaip užsakymo atšaukimas arba panaikinimas iš sąrašo. Sunkumas vidutinis, nes vartotojui pirmus kelis kartus mygtumo paskirtis gali pasirodyti dviprasmiška. \\ \hline

   \hline
    \end{tabular}
\end{center}
\begin{center}
    \begin{tabular}{ |p{3cm}| p{3cm} | p{11cm} | }
	\hline
	Klaidų prevencija & Sunkus & Prekės informacijos redagavimo lange yra mygtukas ,,Išimti prekę". 
					Vartotojui paspaudus jį pasirodo pranešimas klausiantis ,,Ar tikrai norite ištrinti prekę". 
					Paspaudus jį prekė yra pašalinama iš sąrašo.
					Nėra pateikta jokios ,,šiukšledėžės" arba ,,undo" galimybės, todėl vartotojui per klaidą patvirtinus prekės pašalinimą visa svarbi prekės informacija yra pašalinama nesugrįštamai. \\ \hline
	Naudojimo lankstumas ir našumas & Lengvas & Vartotojui nėra suteikiama galimybė naudotis klavišų santrumpom, pelės santumpom, navigacijos šuoliais ar panašiais metodais palengvinančiais darbą.
								Šis pažeidimas įvertintas kaip lengvas, nes programoje nėra daug funkcionalumo todėl tumpiniai daug laiko nesutaupytų. \\ \hline
	Pagalba ir dokumentacija & Lengvas & Sistema vartotojui nesuteikia jokios dokumentacijos, apmokymų ar aprašymų kaip naudotis sistema. 
							Įvertinta kaip lengvas defektas nes funkcionalumo yra nedaug ir jis išmokstamas pačiam bandant naudotis sistema. \\ \hline
   \hline
    \end{tabular}
\end{center}
	\subsection{Pažintinė peržvalga}

\begin{center}
    \begin{tabular}{ |p{4cm}| p{6cm} | p{7cm} | }
    \hline
    Užduoties žingsniai & Ar aiškiai matoma, ką daryti? & Ar suprantamas atsakas \\ \hline 
   \hline
    \end{tabular}
\end{center}
	\subsection{Apibendrinimas}
\section{Andriaus Bentkaus maketo vertinimas}
	\subsection{Euristinis tikrinimas}
\begin{center}
    \begin{tabular}{ |p{3cm}| p{3cm} | p{11cm} | }
    \hline
    Euristika &Defekto sunkumas &Komentaras \\ \hline 
   \hline
    \end{tabular}
\end{center}
	\subsection{Pažintinė peržvalga}
\begin{center}
    \begin{tabular}{ |p{4cm}| p{6cm} | p{7cm} | }
    \hline
    Užduoties žingsniai & Ar aiškiai matoma, ką daryti? & Ar suprantamas atsakas \\ \hline 
   \hline
    \end{tabular}
\end{center}
	\subsection{Apibendrinimas}
\iffalse XXXXXXXXXXXXXXXXXXXXXXXXXXXXXXXXXXXXXXXXXXXXXXXXXXXXXXXXXXXXXXXXXXXXXXXXXXXXXXXXXXXXXXXXXXXXXXXXXXXXXXXXXXXXXXXXXXXXXXXXXXXXXXXXXXXXXXX \fi
\section{Išvados}


\end{document}
